\documentclass{article}
\usepackage{graphicx}
\usepackage{amsmath} % For math formatting


\begin{document}

\title{Fall-23 5304 LecN7 Notes}
\author{Wan}
\date{\today}
\maketitle

\noindent
Topics: The Gram-Schmidt algo and the QR Factorization; Least-squares problems; Applications; Data fitting.\\


\noindent
Math tools: Eigenvalues and Eigencevectors

\section{Least-Squares Systems and Data fitting}
% 这个地方不是很理解,可能之后再回来研究
\subsection*{Problem BG}
1, overdetermined system, more euqations than unknowns, no sol\\
\includegraphics[width=1\linewidth]{7-1}\\
\\
2, Function approximation

3, Goal: Find the best approximation to the system of equations.\\

\subsection*{Geometric Interpretation}

\subsection*{Good illustration: Data fitting}


\pagebreak
\section{Gram-Schmidt}
\subsection{Concept}

\subsection{Classical Gram-Schmidt}

The cost of classical Gram-Schmidt is $2mn^2 - \frac{2}{3}n^3$ flops.\\

\noindent
Verification of Algo:\\
1, A = QR. by checking whether norm(A - Q*R) is a small number.\\
2, Orthogonality: Q is orthogonal. by checking whether $Q^TQ = I$, norm(Q'* Q - eye(n)) should be small.

\subsection{Modified Gram-Schmidt}

Cost of QR by using classical Gram-Schmidt:\\


\subsection*{Cost}



\pagebreak
\section{QR Factorization}
\subsection{BG}
The norm here is referring to the 2-norm.\\
\subsection{Concept}
\includegraphics*[width=1\linewidth]{7-3}
A = QR where Q is an orthogonal matrix and R is an invertible upper triangular matrix. If A is
non singular, then this factorization is unique.\\

% R 是正线上三角

\subsection{Process}
Suppose $A = [c_1, c_2, ... c_n]$ is a full-rank matrix, size $m \times n$.\\
\\
Basically 3 steps: \\
1, find A's orthogonal set $F = [f_1, f_2 ... f_n]$.\\
2, normalize each col of $F$ to get $Q = [q_1, q_2 ... q_n]$.\\
3, find $R$ by $R = Q^TA$.\\
\\
To find $F$, use this formula where $f_1 = c_1$ (based on the orthogonal set): \\
$
f_k = c_k - \frac{c_k * f_1}{\left\lVert f_1\right\rVert^2} f_1
- \frac{c_k * f_2}{\left\lVert f_2\right\rVert^2} f_2
- ...
- \frac{c_k * f_{k-1}}{\left\lVert f_{k-1}\right\rVert^2} f_{k-1}
$ (k $>$ 2)
\\
\\
To find $Q$, normalize each col of $F$ by $q_j = \frac{f_j}{\left\lVert f_j\right\rVert }$, j = 1 to n.\\
At this time, $q_1, q_2 ... q_n$ are orthonormal cols.\\
\\
\\
QR Visulization ($c_j$ can be expressed as a linear combination of $q_j$):\\
\includegraphics*[width=1\linewidth]{7-2}


% 我们可以直接通过Q^TA = R 得出R

\subsection*{Examples}
See topic Notes.

\subsection*{Cost}
lalala

\subsection*{Application}
\includegraphics[width=1\linewidth]{7-4}\\

\end{document}