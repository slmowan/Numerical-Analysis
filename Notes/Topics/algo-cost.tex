\documentclass{article}
\usepackage{graphicx}
\usepackage{amsmath} % For math formatting
\usepackage{amssymb}

\begin{document}

\title{Flops of Some Algos}
\author{Wan}
\date{\today}
\maketitle

\section*{Back-Substitution, operation count}
\includegraphics[width=1\linewidth]{a-1}

\section*{Forward-Substitution, operation count}

\pagebreak
\section*{Gaussian Elimination}
\includegraphics[width=1\linewidth]{a-2}

\noindent
What is the operation count (leading term only) for solving the linear system $Ax = b$
with Gaussian elimination without pivoting?\\
$\frac{1}{2} * n^3$\\

\noindent
What happens when partial pivoting is used?

\pagebreak
\section*{Gauss-Jordan Elimination}
\includegraphics[width=1\linewidth]{a-3}

\pagebreak
\section*{LU decomposition}
$\frac{2}{3} * n^3$.\\
\includegraphics[width=1\linewidth]{a-5}

\pagebreak
\section*{Use LU to solve linear sys, cost}
\includegraphics[width=1\linewidth]{a-4}

\pagebreak
\section*{Cholesky decomposition}


\pagebreak
\section*{Gram-Schmidt}
\includegraphics[width=1\linewidth]{a-6}
Note: this is linear in m (number of rows) and quadratic in n (number
of columns).\\  

\pagebreak
\section*{QR decomposition}
\includegraphics[width=1\linewidth]{a-7}


\pagebreak
\section*{Comparison}
1, Gauss-Jordan is 50 percent more expensive than GE. This additional cost is not worth it in spite of the simplicity
of the algorithm. For this Gauss-Jordan is seldom used in practice. If gauss takes 60 secs to complete then gauss-jordan needs
$60*(1+0.5)=60+30=90$ secs to complete.\\
\\
2, True or false: "Computing the LU factorization of matrix A involves more arithmetic operations
than solving the system Ax = b by Gaussian."\\
False. 
The number of arithmetic operations of LU and Gauss is identical. (just LU involves additional memory to store the factors -
but these are not floating point operations).\\
\\
3, QR is 3 times more expensive than GE(LU decomposition). If gauss takes 60 secs to complete then gauss-jordan needs
$3*60 = 180$ secs to complete.\\






\end{document}