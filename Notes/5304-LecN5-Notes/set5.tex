\documentclass{article}
\usepackage{graphicx}
\usepackage{amsmath} % For math formatting


\begin{document}

\title{Fall-2023 5304 LecN5 Notes}
\author{Wan}
\date{\today}
\maketitle


\section{Question}
BG: Normwise backward error\\
Question: Find smallest perturbation to apply to A, b so that exact solution of perturbed system is y\\
After understanding this, then back to Exer 8


\section{Exer 8}

Comments:
Perturb every entry by a small scalar, 


\section{Topic: LEMMA1, LEMMA2, Theorem1, Theorem2}
\subsection*{Lemma1}
If $\left\lVert E\right\rVert < 1$, then $I-E$ is nonsingular and
$\left\lVert(I-E)^{-1} \right\rVert \leq \frac{1}{1-\left\lVert E\right\rVert }$

\subsection*{Lemma2}
Lemma2 generalizes Lemma1.


\section{Topic: Condtion Number-Definition and Peoperties}


\section{Topic: Norm-based Error Bounds}
\subsection*{Forward err and Backward err}
Conclusion:
1, forward error expression is $\frac{\left\lVert x-y\right\rVert}{\left\lVert x\right\rVert}$ where y is the approximate solution by the perturbation system.\\
\\
2, backward error expression is $\left\lVert r\right\rVert $


\section{Topic Estimating condition number, 5-17}
BG: We need cond num to estimate the illness of a system, $\left\lVert A\right\rVert$ is easy
to compute, but $A^{-1}$ is not. Computing $\left\lVert A^{-1}\right\rVert $ is expensive.\\
Thus, we sometimes just want to know a lower bound for the cond num, which implies the illness
of the system is at least as bad as this bound.\\
There are two ways to do it.

\subsection*{Method1: Select a vector v where norm(v)=1, 5-17}
Assumption (2): \\
$\left\lVert v\right\rVert =1$\\
$\left\lVert Av\right\rVert$ is small. (A is near-singular)\\
\\
\noindent
Conclusion:\\
$\left\lVert v\right\rVert  = \left\lVert A^{-1} Av\right\rVert \leq \left\lVert A^{-1}\right\rVert \left\lVert Av\right\rVert $\\
Thus,\\
$\left\lVert A^{-1}\right\rVert \geq \frac{\left\lVert v\right\rVert}{\left\lVert Av\right\rVert}$\\

\noindent
Steps to solve problem:\\
1, observe given A, find a v, normally there exists linear combination relation in A.\\
2, calculate Av\\
3, Normally take 1-norm (col) or inf-norm(row)

Example:


\subsection*{Method2: Near-sigularity, 5-19}
Conclusion:\\
$cond(A) * \left\lVert A-B\right\rVert \geq \left\lVert A\right\rVert $\\
\\
$cond(A) \geq \frac{\left\lVert A\right\rVert }{\left\lVert A-B\right\rVert }$\\
\\
$\frac{1}{cond(A)} \leq \frac{\left\lVert A\right\rVert }{\left\lVert A-B\right\rVert }$\\
\\
\noindent
Method2 proof:\\

\noindent
Method Desp: If you take A and find that it is close to a singular matrix b,
then $\left\lVert A-B\right\rVert $ will as large as it can, so that $\frac{\left\lVert A\right\rVert }{\left\lVert A-B\right\rVert }$
is small, implying a good lower bound.
then:\\

\section{Esitimating Errs from Residual Norms}
BG: Forward error
Residual norm is a very common way to stop algo.

Proof: used Theorem 1 when E = 0.


\section{Appendix: Component-based Error Bounds}

\end{document}