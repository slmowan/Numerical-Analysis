\documentclass{article}
\usepackage{graphicx}
\usepackage{amsmath} % For math formatting


\begin{document}

\title{Fall-23 5304 LecN7 Notes}
\author{Wan}
\date{\today}
\maketitle

\noindent
Topics: The Gram-Schmidt algo and the QR Factorization; Least-squares problems; Applications; Data fitting.\\


\noindent
Math tools: Eigenvalues and Eigencevectors

\section{Least-Squares Systems and Data fitting}
% 这个地方不是很理解,可能之后再回来研究
\subsection*{Problem BG}
1, overdetermined system, more euqations than unknowns, no sol\\
\includegraphics[width=1\linewidth]{7-1}\\
\\
2, Function approximation

3, Goal: Find the best approximation to the system of equations.\\

\subsection*{Geometric Interpretation}

\subsection*{Good illustration: Data fitting}

\pagebreak
\section{QR Factorization}


\subsection{Concept}
\includegraphics*[width=1\linewidth]{7-3}
A = QR where Q is an orthogonal matrix and R is an invertible upper triangular matrix (diagonal entry elements are all positive).
If A is non-singular, then this factorization is unique.\\

QR Visulization ($c_j$ can be expressed as a linear combination of $q_j$):\\
\includegraphics*[width=1\linewidth]{7-2}

\pagebreak

\section{Using classical Gram-Schmidt to get QR}

\subsection*{Classical Gram-Schmidt}
\medskip
\textbf{Input and Ouput Description}\\
Input: $X = \left[x_1,x_2...x_n\right]$, should be full rank (columns are linearly independent), but not
necessarily sqare.\\

\noindent
Output: Will get Q and R where Q is orthogonal ($Q^TQ = I$), and R is upper triangular.\\

\noindent
Picture:\\
\includegraphics[width=0.5\linewidth]{7-5}

\noindent
\textbf{Process}\\
\includegraphics[width=0.5\linewidth]{7-6}
\includegraphics[width=0.5\linewidth]{7-7}

\pagebreak
\noindent
\textbf{Using Classical Gram-Schmidt to get QR, start from step j}\\
\\
\includegraphics[width=1\linewidth]{7-8}
\includegraphics[width=1\linewidth]{7-9}



\subsection*{Cost}
The cost of classical Gram-Schmidt is $2mn^2 - \frac{2}{3}n^3$ flops.\\

\pagebreak
\section{Using modified Gram-Schmidt to get QR (better)}

\noindent
Verification of Algo:\\
1, A = QR. by checking whether norm(A - Q*R) is a small number.\\
2, Orthogonality: Q is orthogonal. by checking whether $Q^TQ = I$, norm(Q'* Q - eye(n)) should be small.

\subsection{Modified Gram-Schmidt}

Cost of QR by using classical Gram-Schmidt:\\


\subsection*{Cost Analysis}



% R 是正线上三角

% \subsection{Process}
% Suppose $A = [c_1, c_2, ... c_n]$ is a full-rank matrix, size $m \times n$.\\
% \\
% Basically 3 steps: \\
% 1, find A's orthogonal set $F = [f_1, f_2 ... f_n]$.\\
% 2, normalize each col of $F$ to get $Q = [q_1, q_2 ... q_n]$.\\
% 3, find $R$ by $R = Q^TA$.\\
% \\
% To find $F$, use this formula where $f_1 = c_1$ (based on the orthogonal set): \\
% $
% f_k = c_k - \frac{c_k * f_1}{\left\lVert f_1\right\rVert^2} f_1
% - \frac{c_k * f_2}{\left\lVert f_2\right\rVert^2} f_2
% - ...
% - \frac{c_k * f_{k-1}}{\left\lVert f_{k-1}\right\rVert^2} f_{k-1}
% $ (k $>$ 2)
% \\
% \\
% To find $Q$, normalize each col of $F$ by $q_j = \frac{f_j}{\left\lVert f_j\right\rVert }$, j = 1 to n.\\
% At this time, $q_1, q_2 ... q_n$ are orthonormal cols.\\
% \\
% \\


% 我们可以直接通过Q^TA = R 得出R

\subsection*{Examples}
See topic Notes.

\subsection*{Cost}
lalala

\subsection*{Application}
\includegraphics[width=1\linewidth]{7-4}\\

\subsection*{Application1: Use QR to solve least-squares problems}
11-09

unitary matrix perserves the length of a vector.\\

rank-1 update

3 steps:\\
1, transform problem % 利用了unitary的性质
2, gradient 

\end{document}